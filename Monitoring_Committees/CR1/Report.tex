\documentclass[11pt]{article}


\usepackage{bm,amsmath,bbm,amsfonts,nicefrac,latexsym,amsmath,amsfonts,amsbsy,amscd,amsxtra,amsgen,amsopn,bbm,amsthm,amssymb,graphicx,natbib}
\usepackage{fancyhdr}




\title{First Committee Meeting\\\vspace{4mm} \normalsize{Understanding the information content in diverse observations of forest carbon stocks and fluxes for data assimilation and ecological modelling.}}
\date{\normalsize{4$^{th}$ December 2013, \ Room 1L36}}
\author{\normalsize{E. Pinnington}}


\newtheorem{theorem}{Theorem}[section]
\newtheorem*{defn}{Definition}



	
\begin{document}

\maketitle

\subsection*{Background}

I began this PhD having just completed a BSc in Mathematics. My modules this term include taking three examinable MSc courses (Data Assimilation, Numerical Solutions to ODE's and Atmospheric Physics) and one formative course (Computing Techniques and Projects). I am sitting in on Computing Techniques and Projects to improve my programming skills. 

\subsection*{Project Overview}

A large amount of data is currently being gathered that is relevant to the carbon balance of forests. Attempts are also being made to combine this data with models of forest carbon stocks and fluxes (such as DALEC) in a data assimilation (DA) scheme. Currently, however, there are limitations with such schemes as there is a lack of understanding about the additional information provided by different observations. Current DA schemes for ecosystem carbon flux only specify the diagonal elements of the observation error covariance, $R$, which correspond to the individual uncertainties in particular observations. As such, these DA schemes do not specify observation error correlations or covariances, corresponding to the off diagonal elements of $R$. Better understanding the information content and error correlations of carbon balance observations will form one of the main aims of the project. 

This project is a NERC CASE partnership with Forest Research. The deciduous Forest Research site at Alice Holt will be the primary source of observation data. The better specification of $R$ will also hopefully lead to the development of a software tool to help inform Forest Research as to which new set of observations to fund that will best benefit the model and DA scheme.

\subsection*{First Term}

So far this term I have been attending the MSc modules stated previously, completing coursework for all four modules. I have undertaken extra reading to better understand my project, this has included:
\begin{itemize}
\item Extracts from Laura Stewarts Thesis \cite{stewart2008correlated} on techniques used in DA for numerical weather prediction.
\item `Breathing of the Terrestrial Biosphere' \cite{baldocchi2008turner} to better understand the type of observations made of forest carbon stocks and fluxes.
\item Mathew Williams paper `An improved analysis of forest carbon dynamics using data assimilation' \cite{williams2005improved} to see the implementation of the DALEC model in a DA scheme.
\end{itemize}
I have also read \cite{fox2009reflex},\cite{tiedeman2013effect} and \cite{delahaies2013regularization} for extra background.

I attended one RRDP course, `Basic Stats Refresher', which I found useful as there was no statistical content on my Mathematics degree course. I plan to attend more RRDP courses next term. I am attending the departmental seminars to gain a background in meteorology. 

In November I visited the Forest Research site at Alice Holt with three of my supervisors from Reading to meet my CASE supervisor James Morrison. This enabled me to view the type of instruments used in taking measurements of carbon stocks and fluxes. It also allowed myself and the supervisors to discuss the aims and direction of the project with the team at Forest Research. It was decided that we would begin by considering the DALEC-D model (for deciduous forests).

\subsection*{Next Monitoring Committee} 

By our next monitoring committee I will have completed my first term modules and will be studying three new modules (Operational DA, Big Data Visualization and Boundary Layer Meteorology). I will continue to read relevant texts and begin learning how to implement DALEC-D. I will hopefully have had another visit with Forest Research, learning how to use a specific piece of instrumentation. I also intend to have taken more RRDP courses in order to increase my transferable skills.


\bibliography{../../PhD}{}
\bibliographystyle{plain}
\end{document}
