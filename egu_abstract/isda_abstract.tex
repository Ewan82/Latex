\documentclass[11pt]{article}
\usepackage[sort]{natbib}
\usepackage{bm,amsmath,bbm,amsfonts,nicefrac,latexsym,amsmath,amsfonts,amsbsy,amscd,amsxtra,amsgen,amsopn,bbm,amsthm,amssymb,graphicx}
\usepackage{fancyhdr}
\usepackage[margin=1.0in]{geometry}

\title{ISDA Abstract}
\author{Ewan Pinnington}

\newtheorem{theorem}{Theorem}[section]
\newtheorem*{defn}{Definition}


\begin{document}


\maketitle

Forest ecosystems sequester large amounts of carbon-dioxide from the atmosphere and thus help mitigate the effect of anthropogenic induced climate change. For that reason understanding their response to climate change is of great importance. Efforts to implement variational data assimilation routines with models of forest carbon balance have been limited. When data assimilation has been used with these models, background and observation errors have largely been treated as independent and uncorrelated. Correlations in the background error covariance are known to be a key aspect of data assimilation in numerical weather prediction. More recently, it has been shown that accounting for correlated observation errors can considerably improve data assimilation results and forecasts. In this presentation we implement a 4D-Var scheme with a simple model of forest carbon balance, for joint parameter and state estimation and assimilate daily observations of Net Ecosystem $\text{CO}_{2}$ Exchange (NEE) taken at the Alice Holt forest $\text{CO}_2$ flux site in Hampshire, UK. We then investigate the effect of specifying parameter and state correlations in background error statistics and time correlations between observation error statistics. The idea of including correlations in time is new and has not been previously explored in carbon balance model data assimilation. We outline novel methods for creating correlated versions of the background and observation error covariance matrices, using a set of previously postulated dynamical constraints for the background error statistics and a Gaussian correlation function to include time correlations between observation errors. In our experiments we compare the results using our new correlated matrices to those using diagonal representations of these matrices. We found that using the new correlated matrices reduced the root mean square error in the $14$ year forecast of daily NEE by $44\%$ decreasing from $4.22~\text{g~C~m}^{-2}~\text{day}^{-1}$ to $2.38~\text{g~C~m}^{-2}~\text{day}^{-1}$.     

\end{document}