\documentclass[11pt]{article}
\usepackage[sort]{natbib}
\usepackage{bm,amsmath,bbm,amsfonts,nicefrac,latexsym,amsmath,amsfonts,amsbsy,amscd,amsxtra,amsgen,amsopn,bbm,amsthm,amssymb,graphicx}
\usepackage{fancyhdr}
\usepackage[margin=1.0in]{geometry}

\title{Literature Review Plan}
\author{Ewan Pinnington}

\newtheorem{theorem}{Theorem}[section]
\newtheorem*{defn}{Definition}


\begin{document}

\maketitle

\begin{itemize}
\item Importance of forest ecosystems to the carbon cycle and negating human induced climate change. The increasing number of available observations relevant to understanding the carbon balance of forests.

\item Many efforts made to combine observations with models to improve our understanding of forest ecosystems, currently not clear which observations provide the most information. Different types of data assimilation.

\item In NWP many efforts made to understand the information content in different sets of observations. Use data assimilation scheme to assess the impact of different observations.

\item Currently in forest carbon model data assimilation schemes correlations between observation errors and background estimate errors have been ignored. It has been shown in NWP that this can lead to lose of information and unrealistic estimates.

\item Some more stuff!


\end{itemize} 



\end{document}