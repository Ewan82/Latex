\documentclass[11pt]{article}


\usepackage[sort]{natbib}
\usepackage{bm,amsmath,bbm,amsfonts,nicefrac,latexsym,amsmath,amsfonts,amsbsy,amscd,amsxtra,amsgen,amsopn,bbm,amsthm,amssymb,graphicx}
\usepackage{fancyhdr}
\usepackage{caption}
\usepackage{subcaption}
\usepackage[margin=1.0in]{geometry}


\title{Information Content for Observations of Forest Carbon Stocks and Fluxes when Assimilated with the DALEC Carbon Balance Model \\ PLAN}
%\date{\normalsize{3$^{\text{rd}}$ June 2014, \ Room 1L36}}
\author{\normalsize{E. Pinnington}}


\newtheorem{theorem}{Theorem}[section]
\newtheorem*{defn}{Definition}

	
\begin{document}

\maketitle

\section*{Introduction}
Motivation for understanding the information content of observations. Important for knowing when and which observations to take to give you the most information possible about the system.  \cite{baldocchi2008turner, fox2009reflex, richardson2010estimating}

\section*{The DALEC Model}
Background on DALEC model and how it is implemented in DA. State just five carbon pools, if estimating the parameters also state is approximately twenty elements.\cite{williams2005improved, delahaies2013regularization}

\section*{Introduction to Variational Assimilation}
For our information content measures we use a variational assimilation framework. 3DVAR/4DVAR. \cite{lewis2006dynamic}

\section*{Information Content Measures}
A look at what information content measures are currently in use from \cite{rodgers2000inverse, stewart2008correlated}.

\section*{Shannon Information Content}
explanation of what $SIC$ is. \cite{rodgers2000inverse, stewart2008correlated}
\subsection*{$SIC$ for a Single Observation at One Time}
A analytic representation of $SIC$ for obs. at a single time. Show $NEE$'s $SIC$ dependence on temperature, net daily irradiance and $Cf$. Obs. of $NEE$ made in summer months provide more information than those made in winter. 
\subsection*{$SIC$ for Successive Observations over a Time Window}
Show information content when successive observations are added to our assimilation window. Begin with analytical form form successive observations of $Cf$. For $NEE$ seasonal cycle still clear, if assimilation window begins in summer a higher $SIC$ is reached over a shorter period (refer to graph in previous section). A single observation of $NEE$ made in summer provides the same level of information as 14 observations of $NEE$ made in winter.

\section*{Degrees of Freedom for Signal}
Backs up findings using $SIC$. Only observing three carbon pools with $NEE$ does this mean max value $DOFS$ can attain is three? \cite{rodgers2000inverse}

\section*{Conclusions}
Summary of what we have found. When it is best to take obs of carbon stocks and fluxes and which obs to take. Although these experiments carried out for DALEC principles will apply for most other carbon balance models.


\bibliography{../PhD}{}
\bibliographystyle{plain}
\end{document}
