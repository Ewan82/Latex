\documentclass[11pt]{article}
\usepackage[sort]{natbib}
\usepackage{bm,amsmath,bbm,amsfonts,nicefrac,latexsym,amsmath,amsfonts,amsbsy,amscd,amsxtra,amsgen,amsopn,bbm,amsthm,amssymb,graphicx}
\usepackage{fancyhdr}
\usepackage[margin=1.0in]{geometry}

\title{Conference abstract}
\author{Ewan Pinnington}

\newtheorem{theorem}{Theorem}[section]
\newtheorem*{defn}{Definition}


\begin{document}

\maketitle

Forest ecosystems play a large role in removing human emitted carbon-dioxide from the atmosphere and therefore greatly reduce the effect of anthropogenic induced climate change. For that reason understanding their response to climate change is of great important. Measurements of forest carbon balance are routinely made in forests across the world using micrometeorological techniques, with many other relevant observations also being available. Currently, however, the optimal set of observations for understanding the carbon balance of a forest is not known.

In order to address this question we use data assimilation. Data assimilation is a set of mathematical techniques to combine data with models in order to improve our estimate of a system. By combining observations with a model, we can judge which observations have had the greatest effect on improving our models estimate by using information content measures.

In this presentation we use the Data Assimilation Linked Ecosystem Carbon model (DALEC2) in a 4d-Var data assimilation framework for parameter and state estimation. This framework allows us to combine our modelled carbon balance from DALEC2, with observations from the Alice Holt research forest run by Forest Research. In this framework we can analyse the information content in different observations relevant to the carbon balance of a forest, in order to understand which observations are helping to improve our models estimate the most. 


\end{document}