\chapter*{\centering \Large \vspace{-20mm}\Huge Abstract}

%=========================================================
%Abstract goes here

%Forests and terrestrial ecosystems play an important role in the global carbon cycle, removing large amounts of CO\(_2\) from the atmosphere and thus helping to mitigate the effect of human-induced climate change. Land surface carbon uptake is the most uncertain process in current estimates of the global carbon cycle. There is much disagreement on whether forests and terrestrial ecosystems will continue to remove the same percentage of CO\(_2\) from the atmosphere under future emission scenarios. It is therefore important to improve understanding of ecosystem carbon cycle processes in context of a changing climate. 

%In this talk we explore new techniques to improve estimates of ecosystem carbon uptake, with focus on the Alice Holt research forest in Hampshire. We then use these techniques, along with complimentary measurements obtained from an extensive fieldwork campaign, to better understand the effect of selective felling on the carbon dynamics of the Alice Holt forest. The response of ecosystem carbon uptake to land use change and disturbance (e.g. fire, felling, insect outbreak) is another large uncertainty in the global carbon cycle. Our methods represent a novel way to help elucidate this response. 

%In this talk we explore new techniques to improve estimates of ecosystem carbon uptake, with focus on the Alice Holt research forest in Hampshire. We then use these techniques, along with complimentary measurements obtained from an extensive fieldwork campaign, to better understand the effect of disturbance from selective felling on the carbon dynamics of the Alice Holt forest. Our methods represent a novel way to help elucidate the response of the forest to disturbance and changing climate.

%Forests play an important role in the global carbon cycle, removing large amounts of CO\(_2\) from the atmosphere and thus helping to mitigate the effect of human-induced climate change. The state of the global carbon cycle in the IPCC AR5 suggests that the land surface is the most uncertain component of the global carbon cycle.  The response of ecosystem carbon uptake to land use change and disturbance (e.g. fire, felling, insect outbreak) is a large component of this uncertainty. A main cause for the high level of uncertainty in terrestrial carbon balance predictions arise from significant gaps in current direct observations and poor parameterisations or missing processes in current modelled predictions. The mathematical technique of data assimilation presents a method for combing incomplete observational records with modelled predictions in order to find the best estimate for the state and parameter variables of a system. Data assimilation has risen to prominence in the field of numerical weather prediction where it has contributed to significant improvements in model forecasts. More recently data assimilation has been used to improve our knowledge of land surface carbon cycle processes. This field is still relatively new and underdeveloped in comparison with numerical weather prediction.

%In this thesis we aim to develop novel data assimilation techniques for the terrestrial carbon cycle. There are three main areas which we address; Understanding the information content in observations for data assimilation, Improving the characterisation of uncertainties in both prior model and observational estimates and using data assimilation to understand the effect of disturbance on the carbon dynamics of a managed woodland.

%It is important to understand which observations can add most information to data assimilation schemes so that choices for field campaigns can be made more sophisticatedly and the most uncertain components of land surface models constrained. In order to better understand information content and improve data assimilation results it is also imperative that we improve the characterisation and estimates of errors in our data assimilation schemes. Using these novel data assimilation techniques along with supplementary observations from a field work campaign we attempt to improve understanding of the effect of selective felling on forest carbon dynamics.         

%Additionally, there is much disagreement on whether forests and terrestrial ecosystems will continue to remove the same proportion of CO\(_2\) from the atmosphere under future climate regimes. It is therefore important to improve our understanding of ecosystem carbon cycle processes in the context of a changing climate.

Forests play an important role in the global carbon cycle, removing large amounts of CO\(_2\) from the atmosphere and thus helping to mitigate the effect of human-induced climate change. Land surface carbon uptake and its response to disturbance (e.g. fire, felling, insect outbreak) are the most uncertain component in the global carbon cycle. A main cause for the high level of uncertainty in terrestrial carbon balance predictions arise from significant gaps in current direct observations and poor parameterisations or missing processes in current modelled predictions. The mathematical technique of data assimilation presents a method for combing incomplete observational records with modelled predictions in order to find the best estimate for the state and parameter variables of a system. Data assimilation has risen to prominence in the field of numerical weather prediction where it has contributed to significant improvements in model forecasts. More recently data assimilation has been used to improve our knowledge of land surface carbon cycle processes. This field is still relatively new and underdeveloped in comparison with numerical weather prediction.

In this thesis we aim to develop novel data assimilation techniques for the terrestrial carbon cycle. There are three main areas which we address: understanding the information content in observations for data assimilation, improving the characterisation of uncertainties in both prior model and observational estimates and using data assimilation to understand the effect of disturbance on the carbon dynamics of a managed woodland. It is important to understand which observations add most information to data assimilation schemes in order to further understanding of unconstrained systems. We show how information content can vary temporally and with different characterisations of errors. We outline new representations of error for ecosystem carbon balance data assimilation schemes (specifically correlations between errors) and show how these can significantly improve data assimilation results. We then use these novel techniques in addition to supplementary observations obtained from a fieldwork campaign to investigate the effect of disturbance from selective felling on forest carbon dynamics.     

%It is important to understand which observations can add most information to data assimilation schemes so that choices for field campaigns can be made more sophisticatedly and the most uncertain components of land surface models constrained. In order to better understand information content and improve data assimilation results it is also imperative that we improve the characterisation and estimates of errors in our data assimilation schemes. Using these novel data assimilation techniques along with supplementary observations from a field work campaign we attempt to improve understanding of the effect of selective felling on forest carbon dynamics.   
