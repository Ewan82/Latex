%\chapter{Methods and tools}
\section{An technical overview of climate modelling}
\label{chap:method_techover_climmod}
From a technical point of view, and detail how climate models can vary. AM's, AOGCMs and ESMs. Make sure to cite \cite{Meehl2007} and \cite{Hibbard2007}.
\section{CMIP5 models}
\label{chap:method_CMIP5}
Introduce the CMIP5 models. Make sure to cite summary of experimental design of CMIP5: \cite{Taylor2007}. Explain their resolution, and other important aspects introduced. All models listed in Table \ref{tab:models}.
\begin{table}
  %  \footnotesize
  \centering

  \caption{\label{tab:models}Modelling data -- the CMIP5 multi-model ensemble, and observationally derived data -- the ERA-Interim reanalysis in addition for comparison.}
  %\vspace{6pt}
  \begin{tabular}{ccccccc} \hline \hline
    \textbf{Model} & \textbf{Resolution} & \multicolumn{4}{c}{\textbf{Ensemble Size}} \\
     & ($\textrm{lon}$\times$\textrm{lat (Spectral), levels}$) & \footnotesize{Historical} & \footnotesize{AMIP} & \footnotesize{RCP4.5} & \footnotesize{RCP8.5} \\ \hline
    ACCESS1.0 & $1.875^{\circ}\times1.25^{\circ}$, L38 & 1 & 1 & 1 & 1 \\
    ACCESS1.3 & $1.875^{\circ}\times1.25^{\circ}$, L38 & 1 & 1 & 1 & 1 \\ \hline
    BCC-CSM1.1 & $2.813^{\circ}\times2.813^{\circ}$ (T42), L26 & 3 & 3 & 1 & 1 \\
    BCC-CSM1.1(m) & $1^{\circ}\times1^{\circ}$ (T106), L26 & 3 & 3 & 1 & 1 \\ \hline
    BNU-ESM & $2.813^{\circ}\times2.813^{\circ}$ (T42), L26 & 1 & 1 & 1 & 1 \\ \hline
    CanESM2 & $2.813^{\circ}\times2.813^{\circ}$ (T63), L35 & 5 & - & 1 & 1 \\ \hline
    CCSM4 & $1.25^{\circ}\times0.9^{\circ}$, L26 & 1 & 4 & 1 & 1 \\ \hline
    CMCC-CM & $0.75^{\circ}\times0.75^{\circ}$ (T159), L31 & 1 & 3 & 1 & 1 \\ \hline
    CNRM-CM5 & $1.406^{\circ}\times1.412^{\circ}$, L31 & 10 & 1 & 1 & 1 \\ \hline
    CSIRO-Mk3.6.0 & $1.875^{\circ}\times1.875^{\circ}$ (T63), L18 & 10 & 10 & 10 & 10 \\ \hline
    EC-EARTH & $1.125^{\circ}\times1.125^{\circ}$ (T159), L62 & 1 & 1 & 2 & 3 \\ \hline
    FGOALS-g2 & $2.8^{\circ}\times2.8^{\circ}$, L26 & 1 & 1 & 1 & 1 \\ \hline
    FGOALS-s2 & $2.81^{\circ}\times1.66^{\circ}$ (R42), L26 & 3 & 3 & 3 & 3 \\ \hline
    GFDL-CM3 & $2.5^{\circ}\times2.0^{\circ}$ (C48), L48 & 5 & - & 3 & 1 \\
    GFDL-ESM2G & $2.5^{\circ}\times2.0^{\circ}$, L24 & 1 & - & 1 & 1 \\
    GFDL-ESM2M & $2.5^{\circ}\times2.0^{\circ}$, L24 & 1 & - & 1 & 1 \\ \hline
    HadGEM2-A & $1.875^{\circ}\times1.25^{\circ}$, L38 & - & 7 & - & - \\
    HadGEM2-CC & $1.875^{\circ}\times1.25^{\circ}$, L60 & 2 & - & 1 & 2 \\
    HadGEM2-ES & $1.875^{\circ}\times1.25^{\circ}$, L38 & 1 & - & 1 & 1 \\ \hline
    INM-CM4 & $2.0^{\circ}\times1.5^{\circ}$, L21 & 1 & 1 & 1 & 1 \\ \hline
    IPSL-CM5A-LR & $3.75^{\circ}\times1.875^{\circ}$, L39 & 4 & 6 & 4 & 4 \\
    IPSL-CM5A-MR & $2.5^{\circ}\times1.25^{\circ}$, L39 & 1 & 1 & 1 & 1 \\
    IPSL-CM5B-LR & $3.75^{\circ}\times1.875^{\circ}$, L39 & 2 & 1 & 1 & 1 \\ \hline
    MIROC-ESM & $2.8125^{\circ}\times2.8125^{\circ}$ (T42), L80 & 3 & - & 1 & 1 \\
    MIROC-ESM-CHEM & $2.8125^{\circ}\times2.8125^{\circ}$ (T42), L80 & 1 & - & 1 & 1 \\
    MIROC5 & $1.406^{\circ}\times1.406^{\circ}$ (T85), L40 & 1 & 2 & 1 & 1 \\ \hline
    MPI-ESM-LR & $1.875^{\circ}\times1.875^{\circ}$ (T63), L47 & 3 & 3 & 3 & 3 \\
    MPI-ESM-MR & $1.875^{\circ}\times1.875^{\circ}$ (T63), L95 & 3 & 3 & 1 & 1 \\ \hline
    MRI-CGCM3 & $1.125^{\circ}\times1.125^{\circ}$ (T159), L48 & 3 & 5 & 1 & 1 \\ \hline
    NorESM1-M & $2.5^{\circ}\times1.9^{\circ}$, L26 & 3 & 3 & 1 & 1 \\ \hline \hline
    \multicolumn{2}{r}{\textbf{Total ensemble members}} & \textbf{76} & \textbf{64} & \textbf{48} & \textbf{48} \\
    \multicolumn{2}{r}{\textbf{Total ensemble models}} & \textbf{29} & \textbf{22} & \textbf{29} & \textbf{29} \\ \hline \hline
    \textbf{Reanalysis} & \textbf{Resolution} & \multicolumn{4}{c}{\textbf{Data Assimilation}} \\ \hline
    ERA-Interim & $0.75^{\circ}\times0.75^{\circ}$ (T255), L60 & \multicolumn{4}{c}{4DVAR} \\
    GPCP & $2.5^{\circ}\times2.5^{\circ}$ & \multicolumn{4}{c}{\footnotesize{\cite{Adler2003}, \cite{Huffman2009}}} \\
    \hline \hline
  \end{tabular}
\end{table}


\section{Reanalysis and GPCP data}
\label{chap:method_obsdata}
Explain why reanalyses are the best way to assess biases in climate models with respect to the general circulation. Take observations and use data assimilation system and its forecast model to grid the observations and make them consistent across space and time. Explain why I have chosen ERA-Interim. ERA-Interim citation \cite{Dee2011}
Explain why using GPCP and not reanalysis for precipitation. GPCP: The currently operational procedure is described in \cite{Adler2003} and \cite{Huffman2009} and has been used to produce the GPCP Version 2.2 Combined Precipitation Data Set, covering the period January 1979 through the present (with some delay). The primary product in the Version 2.2 dataset is a combined observation-only dataset, that is, a gridded analysis based on gauge measurements and satellite estimates of rainfall.
\section{TRACK}
\label{chap:method_TRACK}
\subsection{Tracking}
Introduce TRACK.
Explain why I have only tracked in 850hPa vorticity (referring back to \ref{chap:intro_lit_methods}).
\subsection{Statistics}
Introduce TRACK statistics.
\section{Multi-model ensemble methodologies}
\label{chap:method_means}
Explain the process of calculating ensemble means, multi-model means and stippling to be used in the following chapters.
\section{Processes involved in calculating global circulation diagnostics}
\label{chap:method_globalcirc}
monthly means
Eady
geopot methodology
boxes for zonal means
