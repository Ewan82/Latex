%Thesis aims and outline

The primary aim of this thesis is the development of data assimilation techniques for the terrestrial carbon cycle, in order to aid progress in this field of research. We focus on implementing novel data assimilation methods with a simple model of ecosystem carbon balance in order to address three key areas:

\begin{enumerate}
\item \textit{Investigating the information content in distinct carbon balance observations for data assimilation}

It is important to understand which observations provide most information to data assimilation schemes. We investigate the different relative levels of information for observations relevant to ecosystem carbon balance through novel applications of information content metrics.

\item \textit{Improving the representation of prior and observational errors in carbon cycle data assimilation}

Currently the specification of both prior and observational errors for the carbon cycle have been simplistic. We seek to improve this representation by investigating the role of correlations between errors for both prior estimates and observations. We judge the effect of including these error correlations on data assimilation results.
  
\item \textit{Using data assimilation to understand the effect of disturbance on forest carbon dynamics}

The effect of disturbance (e.g. fire, felling, insect outbreak) on ecosystem carbon dynamics is one of the least understood components of the global carbon cycle. We investigate the effect of selective felling on forest carbon uptake using novel data assimilation techniques.
\end{enumerate}

From this point onwards the thesis is structured as follows:

\begin{itemize}
\item \textbf{Chapter~\ref{chap:litrev}} introduces the concept of data assimilation and relevant methods. In particular applications of data assimilation to the terrestrial carbon cycle are discussed. We highlight some of the current issues faced and areas for future development.  

\item \textbf{Chapter~\ref{chap:data}} provides an explanation of the Data Assimilation Linked Ecosystem Carbon (DALEC and DALEC2) models used throughout the thesis. The fieldwork campaign conducted during this PhD project is outlined along with additional flux tower data from the Alice Holt research site (Hampshire, UK).

\item \textbf{Chapter~\ref{chap:info_con}} explores the first aim of the thesis. The DALEC and DALEC2 model are are used in a set of information content experiments, with novel applications of information content metrics, in order to better understand the relative levels of information from different observations and how information content might vary in time and with different characterisations of errors.

\item \textbf{Chapter~\ref{chap:error_corrs}} introduces a fully tested data assimilation scheme with the DALEC2 model and uses this to address the second aim of the thesis. The role of prior and observation error correlations are investigated in a set of data assimilation experiments to understand their contribution to improving a model forecast of forest carbon uptake.

\item \textbf{Chapter~\ref{chap:disturbance}} uses the techniques developed in chapter~\ref{chap:error_corrs} along with supplementary observations from the fieldwork campaign outlined in chapter~\ref{chap:data} to address the third aim of the thesis.

\item \textbf{Chapter~\ref{chap:conclusion}} presents the results of the thesis as a whole and discusses opportunities for future work.    

\end{itemize}
     
%Figure Example:

%\begin{figure}[htbp]
%\centering
%\includegraphics[trim=5pc 20pc 5pc 20pc, clip=true, width=0.8\textwidth]{chapter/chapter2/ch2fig1}
%\caption{\small Annual anomalies of global land-surface air temperature ($^\circ$C), 1850 to 2005 from several datasets. The smooth curves show decadal variations. Figure 3.1 from Climate Change 2007: The Physical Science Basis. Contribution of Working Group I to the Fourth Assessment Report of the Intergovernmental Panel on Climate Change.}
%\label{ch2fig1}
%\end{figure}
