%Chapter 1

This thesis has explored data assimilation for the terrestrial carbon cycle. The state of the global carbon cycle in the IPCC AR5 suggests that the land surface is the most uncertain component of the global carbon cycle. The response of ecosystem carbon uptake to land use change and disturbance (e.g. fire, felling, insect outbreak) is a large component of this uncertainty. The uncertainties in land surface carbon cycling processes are largely due to gaps in direct observations and poor parameterisations of model processes. Data assimilation provides methods to improve current estimates by combining observations with prior model estimates. In order to improve data assimilation results it is important that we add the most possible information about a system to data assimilation schemes. The equates to large amounts and more accurate information in, equals better information out. This could mean new observations with high levels of information for constraining poorly understood processes, or better characterisation of prior model and observational errors. Both the optimal set of observations and appropriate representation of error in data assimilation for the carbon cycle are not well understood. Based on this and knowledge of other components of uncertainty three key areas of research were identified in chapter~\ref{chap:intro}:
\begin{enumerate}
\item \textit{Investigating the information content in distinct carbon balance observations for data assimilation}

\item \textit{Improving the representation of prior and observational errors in carbon cycle data assimilation}

\item \textit{Using data assimilation to understand the effect of disturbance on forest carbon dynamics}
\end{enumerate}
The following sections will address these points in turn based on the work presented in this thesis in chapter~\ref{chap:info_con}--\ref{chap:disturbance}.

\section{Investigating the information content in distinct carbon balance observations for data assimilation}

In chapter~\ref{chap:info_con} we used both the DALEC1 and DALEC2 models of ecosystem carbon balance in a series of information content experiments. We calculated the tangent linear model of DALEC1 analytically by hand so that we could make novel applications of information content metrics relying on the adjoint of the model code. For the more complex and nonlinear DALEC2 joint state and parameter estimation case the tangent linear model was calculated using automatic differentiation. From the information content experiments in this chapter we deduced the following conclusions:
\begin{itemize}
\item For both the DALEC1 and DALEC2 models we found our system was observable for the available observations of NEE. This means that for data assimilation we can construct a locally unique solution from observational information alone. This is important as it give us confidence in subsequent experimental results relying on NEE as the main source for observational information.
\item There was a strong temporal variation in information content for observations of NEE, with observations made at times of higher temperatures having higher information content. For deciduous ecosystems observations of NEE made at times of leaf-on and leaf-off have higher influence in the assimilation as these act to help constrain the phenology of the model. 
\item Including an increasing correlation between NEE observation errors in time reduced the information content in the assimilated observations. 
\item It was clear from these experiments that in order to further improve understanding of the information content in observations it is important to improve estimates and representations of uncertainty for both prior model predictions and observations.
\end{itemize}

\section{Improving the representation of prior and observational errors in carbon cycle data assimilation}

In chapter~\ref{chap:error_corrs} we implemented and tested a 4D-Var data assimilation scheme with the DALEC2 model. We then used this system to investigate the effect of including correlations between prior model and NEE observation errors. In each experiment we assimilated a single year of NEE observations (1999) and then ran a 14 year forecast (2000-2014) of NEE. From this work our conclusions were:
\begin{itemize}
\item Including correlations in the background error covariance matrix significantly improved the model forecast after assimilation. Correlations in the observation error covariance matrix between NEE observation errors in time had much less of an effect on results. However, we expect these correlations will have more impact when assimilating more than one data stream or assimilating observations of NEE at a finer temporal resolution.
\item When correlations were included in both the background error covariance matrix and the observation error covariance matrix we found the best model forecast results. In this case the forecast root-mean-square error was reduced by a significant $44\% $ in comparison to when a diagonal background and observation error covariance matrix were used in assimilation.
\end{itemize} 


\section{Using data assimilation to understand the effect of disturbance on forest carbon dynamics}

In chapter~\ref{chap:disturbance} we again utilised the 4D-Var data assimilation scheme with DALEC2 outlined in chapter~\ref{chap:error_corrs} along with a set of observations taken on a fieldwork campaign to investigate the effect of selective felling on the carbon dynamics of the Alice Holt forest. We conducted a data-denial experiments using all the available observations to understand their effect on the modelled predicted effect of disturbance. We also propose novel observation operators allowing for the assimilation of daytime and nighttime NEE observations with a daily time-step model. The main results in this chapter were as follows:
\begin{itemize}
\item The proposed observation operators allow our model to accurately predict daytime and nighttime NEE and negate the need for any model modification. 
\item We find no change to the net ecosystem carbon uptake after felling when approximately 46\% of trees were removed from the area of interest.
\item Our most confident modelled estimate (when all available data is assimilated) suggests this unchanged carbon uptake is due to GPP being the main driver for both autotrophic and heterotrophic respiration, so that even with reduced GPP post-disturbance the same NEE can occur due to significant reductions in total ecosystem respiration. We find different conclusions if we base our conclusions solely on the assimilation of NEE observations, highlighting the need for caution not to over-interpret results when assimilating only this variable. 
\end{itemize}

\section{Future work}

ROUGH DRAFT

The continued application of IC meausres is important to better understand where to base efforts in future observation campaigns. However, first it is maybe more important to improve both estimates of uncertainty for prior model and observational estimates and so that results from IC experiements can be as accurate as possible. It would also be interesting to use a more complex model of ecosystem carobn dynamics where we could judge the impact of more novel meausreents such as stem respiration or sap flow rate.

We have shown that including a more sophisticated respresentation of error in data assimilation schemes can be of great benefit to results. It is imporant that we continue to improve the representation of uncertainty in data assimilation schemes. In relation to the experiments carried out in this thesis its is clear that a more diagnostic tool for the specification of observation erorr correlations is important. One possiblity for this would be to use a method such as the desrozier diagnostic to statistically estimate the error covariance structure for observations. In order to diagnose correlations in time the desroziers diagnostic would have to be expanded. This would be possible be re-writting it as...

SHOW EQUATIONS HERE

The results we find for the effect of disturbance are possible suprising however do support measurements based campaigns analysing soil microbial communities after felling has occured. This experiement would be good to carry out again after collecting a few more years of data to understand the revoery of leaf area and long term response to disturbance. If possible it would be extremely beneficial to also set up soil respiration chambers on both the thinned and unthinned sides of the forest to improve the constraint on constituent processes.