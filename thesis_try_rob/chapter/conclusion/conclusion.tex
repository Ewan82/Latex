%Chapter 1

This thesis has explored data assimilation for the terrestrial carbon cycle. The state of the global carbon cycle in the IPCC AR5 suggests that the land surface is the most uncertain component of the global carbon cycle. The response of ecosystem carbon uptake to land use change and disturbance (e.g. fire, felling, insect outbreak) is a large component of this uncertainty. The uncertainties in land surface carbon cycling processes are largely due to gaps in direct observations and poor parameterisations of model processes. Data assimilation provides methods to improve current estimates by combining observations with prior model estimates. In order to improve data assimilation results it is important that we add the most possible information about a system to data assimilation schemes. The equates to large amounts and more accurate information in, equals better information out. This could mean new observations with high levels of information for constraining poorly understood processes, or better characterisation of prior model and observational errors. Both the optimal set of observations and appropriate representation of error in data assimilation for the carbon cycle are not well understood. Based on this and knowledge of other components of uncertainty three key areas of research were identified in chapter~\ref{chap:intro}:
\begin{enumerate}
\item \textit{Investigating the information content in distinct carbon balance observations for data assimilation}

\item \textit{Improving the representation of prior and observational errors in carbon cycle data assimilation}

\item \textit{Using data assimilation to understand the effect of disturbance on forest carbon dynamics}
\end{enumerate}
The following sections will address these points in turn based on the work presented in this thesis in chapter~\ref{chap:info_con}--\ref{chap:disturbance}.

\section{Investigating the information content in distinct carbon balance observations for data assimilation}

In chapter~\ref{chap:info_con} we used both the DALEC1 and DALEC2 models of ecosystem carbon balance in a series of information content experiments. We calculated the tangent linear model of DALEC1 analytically by hand so that we could make novel applications of information content metrics relying on the adjoint of the model code. For the more complex and nonlinear DALEC2 joint state and parameter estimation case the tangent linear model was calculated using automatic differentiation. From the information content experiments in this chapter we deduced the following conclusions:
\begin{itemize}
\item For both the DALEC1 and DALEC2 models we found our system was observable for the available observations of NEE. This means that for data assimilation we can construct a locally unique solution from observational information alone. This is important as it give us confidence in subsequent experimental results relying on NEE as the main source for observational information.
\item There was a strong temporal variation in information content for observations of NEE, with observations made at times of higher temperatures having higher information content. For deciduous ecosystems observations of NEE made at times of leaf-on and leaf-off have higher influence in the assimilation as these act to help constrain the phenology of the model. 
\item Including an increasing correlation between NEE observation errors in time reduced the information content in the assimilated observations. 
\item It was clear from these experiments that in order to further improve understanding of the information content in observations it is important to improve estimates and representations of uncertainty for both prior model predictions and observations.
\end{itemize}

\section{Improving the representation of prior and observational errors in carbon cycle data assimilation}

In chapter~\ref{chap:error_corrs} we implemented and tested a 4D-Var data assimilation scheme with the DALEC2 model. We then used this system to investigate the effect of including correlations between prior model and NEE observation errors. In each experiment we assimilated a single year of NEE observations (1999) and then ran a 14 year forecast (2000-2014) of NEE. From this work our conclusions were:
\begin{itemize}
\item Including correlations in the background error covariance matrix significantly improved the model forecast after assimilation. Correlations in the observation error covariance matrix between NEE observation errors in time had much less of an effect on results. However, we expect these correlations will have more impact when assimilating more than one data stream or assimilating observations of NEE at a finer temporal resolution.
\item When correlations were included in both the background error covariance matrix and the observation error covariance matrix we found the best model forecast results. In this case the forecast root-mean-square error was reduced by a significant $44\% $ in comparison to when a diagonal background and observation error covariance matrix were used in assimilation.
\end{itemize} 


\section{Using data assimilation to understand the effect of disturbance on forest carbon dynamics}

In chapter~\ref{chap:disturbance} we...



 it is important to understand which observations provide the most information for constraining posterior model estimates. It is also imporant that we add the most information     


It was clear from the first results chapter that in order to better understand the information content in observatnios it is important to improve estimates and representations of uncertainty for both prior modelled estimates and observations.

In the next chapter this is what we did...