%Chapter 1

This thesis has explored data assimilation for the terrestrial carbon cycle. The current understanding of the global carbon cycle in the IPCC AR5 suggests that the land surface is the most uncertain component. The response of ecosystem carbon uptake to land use change and disturbance (e.g. fire, felling, insect outbreak) is a large component of this uncertainty. The uncertainties in land surface carbon cycling processes are largely due to gaps in direct observations and poor parameterisations of model processes. Data assimilation provides methods to improve current estimates by combining observations with prior model estimates. In order to improve data assimilation results it is important that we include as much information as possible about a system. This could mean new observations with high levels of information for constraining poorly understood processes, or better characterisations of prior model and observational errors. Both the optimal set of observations and appropriate representation of error in data assimilation for the carbon cycle are not well understood. Based on this and knowledge of other components of uncertainty three key areas of research were identified in Chapter~\ref{chap:intro}:
\begin{enumerate}
\item \textit{Investigating the information content in distinct carbon balance observations}

\item \textit{Improving the representation of prior and observational errors in carbon cycle data assimilation}

\item \textit{Using data assimilation to understand the effect of disturbance on forest carbon dynamics}
\end{enumerate}
The following sections will address these points in turn based on the work presented in this thesis in Chapter~\ref{chap:info_con}--\ref{chap:disturbance}.

\section{Investigating the information content in distinct carbon balance observations}

In chapter~\ref{chap:info_con} we used both the DALEC1 and DALEC2 models of ecosystem carbon balance in a series of information content experiments. We calculated the tangent linear and adjoint model of DALEC1 analytically by hand so that we could implement measures of information content relying on the adjoint of the model code. For the larger DALEC2 joint state and parameter estimation case the tangent linear and adjoint model was calculated using automatic differentiation. From the information content experiments in chapter~\ref{chap:info_con} we deduced the following conclusions:
\begin{itemize}
\item For both the DALEC1 and DALEC2 models we found our system was observable for the available observations of NEE. This means that for data assimilation we can construct a locally unique solution from observational information alone. This is important as it gives us confidence in subsequent experimental results relying on NEE as the main source for observational information.
\item There was a strong temporal variation in information content for observations of NEE, with observations made at times of higher temperatures having higher information content. For deciduous ecosystems, observations of NEE made at times of leaf-on and leaf-off have higher influence in the assimilation as these act to help constrain the phenology of the model. 
\item Including an increasing correlation between NEE observation errors in time reduced the information content in the assimilated observations. 
\item It was clear from these experiments that in order to further improve understanding of the information content in observations it is important to improve estimates and representations of uncertainty for both prior model predictions and observations.
\end{itemize}

\section{Improving the representation of prior and observational errors in carbon cycle data assimilation}

In chapter~\ref{chap:error_corrs} we implemented and tested a 4D-Var data assimilation scheme with the DALEC2 model. We then used this system to investigate the effect of including correlations between prior model errors and between NEE observation errors. In each experiment we assimilated a single year of NEE observations (1999) and then ran a 14 year forecast (2000-2014). From this work our conclusions were:
\begin{itemize}
\item Including correlations in the background error covariance matrix significantly improved the model forecast after assimilation. Correlations in the observation error covariance matrix between NEE observation errors also had a positive, but much smaller, effect on results. However, we expect these correlations will have more impact when assimilating more than one data stream or assimilating observations of NEE at a finer temporal resolution. This is because here correlations will be of much greater magnitude than those included in our experiments.
\item When correlations were included in both the background error covariance matrix and the observation error covariance matrix we found the best model forecast results. In this case the forecast root-mean-square error was reduced by a significant $44\% $ in comparison to the use of a diagonal background and observation error covariance matrix.
\end{itemize} 


\section{Using data assimilation to understand the effect of disturbance on forest carbon dynamics}

In chapter~\ref{chap:disturbance} we again utilised the 4D-Var data assimilation scheme with DALEC2 outlined in chapter~\ref{chap:error_corrs} along with a set of observations taken on a fieldwork campaign to investigate the effect of selective felling on the carbon dynamics of the Alice Holt forest. We conducted a data-denial experiments using all the available observations to understand their impact on the model predicted effect of disturbance. We also proposed novel observation operators which facilitate the assimilation of daytime and nighttime NEE observations with a daily time-step model. The main results in chapter~\ref{chap:disturbance} were as follows:
\begin{itemize}
\item The proposed observation operators allow our model to accurately predict daytime and nighttime NEE and negate the need for any model modification. 
\item We find no change to the net ecosystem carbon uptake after felling when approximately 46\% of trees were removed from the area of interest.
\item Our most confident modelled estimate (when all available data is assimilated) suggests this unchanged carbon uptake is due to GPP being the main driver for both autotrophic and heterotrophic respiration, so that even with reduced GPP post-disturbance the same NEE can occur due to significant reductions in total ecosystem respiration. We find different conclusions when assimilating only NEE observations, which highlights the need for caution when interpreting results based solely on the assimilation of this variable. 
\end{itemize}

\section{Future work}

The continued application of information content measures is important to better understand where to direct efforts in future observation campaigns. For example, by applying these measures to a model of ecosystem carbon dynamics with a more sophisticated representation of respiration processes, we could judge the impact of new measurements such as stem respiration. Efforts should also be made to continue to improve estimates of uncertainty for both prior model predictions and observations. This will ensure that results from information content experiments can be as accurate as possible. 

We have shown that including a more sophisticated representation of error in data assimilation schemes can be of great benefit to results. Further investigation to improve the representation of uncertainty in data assimilation schemes would be advantageous. In relation to the experiments carried out in this thesis it is clear that a more diagnostic tool for the specification of observation error correlations is needed. One possibility for this would be to use the \citet{desroziers2005diagnosis} diagnostic to statistically estimate the error covariance structure of assimilated observations. In order to diagnose correlations in time (similar to those specified in chapter~\ref{chap:error_corrs}) the \citet{desroziers2005diagnosis} diagnostic would have to be expanded. The \citet{desroziers2005diagnosis} diagnostic estimates the observation error covariance matrix as
\begin{equation}
\textbf{R} = \mathbb{E}[(\textbf{y}-\textbf{h}(\textbf{x}_a))(\textbf{y}-\textbf{h}(\textbf{x}_b))^{T}]
\end{equation}
this could be re-written as
\begin{equation}
\hat{\textbf{R}} = \mathbb{E}[(\hat{\textbf{y}}-\hat{\textbf{h}}(\textbf{x}_a))(\hat{\textbf{y}}-\hat{\textbf{h}}(\textbf{x}_b))^{T}]
\end{equation}
to estimate time correlations in the observation error covariance matrix. This would require that an ensemble of 4D-Var data assimilations were run with perturbed prior model estimates and perturbed observations. We could then retrieve a statistical estimate of the temporal correlation structure for the observation error covariance matrix. It is also possible to account for model error when the matrix \(\hat{\textbf{R}}\) is specified in this way \citep{Howes2017}. 

It will also be useful to test these new covariance matrices with included correlations at other research sites and with larger model implementations. It will be important to test if global estimates can be improved with improved representations of uncertainty. These larger scale applications could also consider the effect of spatial correlations between carbon balance observation errors.

The results we find for the short term effect of disturbance do support ecological measurement campaigns that have analysed soil microbial communities after selective felling events. In order to understand the long term effects of disturbance on the Alice Holt forest, the experiment could be repeated after collecting a few more years of data. This would also give us an insight into the recovery of leaf area. If possible, it would also be extremely beneficial to set up soil respiration chambers on both the thinned and unthinned sides of the forest, observations from which could improve constraint on the constituent processes.

Our findings highlight the need to improve current characterisations of uncertainty in carbon cycle data assimilation schemes. This should ultimately lead to improved modelled forecasts of land surface carbon uptake which would better constrain what is currently a key uncertainty in the global carbon cycle. From Chapter~\ref{chap:info_con} we also see that improving representations of errors will help us to better understand which observations will give us the most information to improve current estimates. 

The search for novel applications of data assimilation to the terrestrial carbon cycle should continue. The application of techniques developed in Chapter~\ref{chap:disturbance} to investigate disturbance on terrestrial carbon dynamics could be widened to other ecosystems and disturbance types. This should lead to a better understanding of the forests globally. 
