\documentclass[11pt]{article}
\usepackage[sort]{natbib}
\usepackage{bm,amsmath,bbm,amsfonts,nicefrac,latexsym,amsmath,amsfonts,amsbsy,amscd,amsxtra,amsgen,amsopn,bbm,amsthm,amssymb,graphicx}
\usepackage{fancyhdr}
\usepackage[margin=0.8in]{geometry}
\bibliographystyle{abbrvnat}
\usepackage{bibentry}
\bibliographystyle{plainnat}


\nobibliography*

\title{\vspace{-2cm}Statement of research interests}
\author{Ewan Pinnington}
\date{}

\newtheorem{theorem}{Theorem}[section]
\newtheorem*{defn}{Definition}


\begin{document}

\maketitle
\vspace{-.5cm}
My research interests are in modelling and data assimilation for the terrestrial carbon cycle. Data assimilation is a mathematical technique for combining observations with a mathematical model, in order to improve our knowledge of a system. This technique requires us to quantify the uncertainty in observations and the uncertainty in prior model predictions. Much of my work has been to improve this representation of uncertainty in data assimilation schemes, to both improve the accuracy of posterior model estimates and forecasts of forest carbon balance. The development of data assimilation techniques and comparison of modelled and observed estimates for the terrestrial carbon cycle is something I would like to continue to pursue. The project assessing the sensitivity of North American ecosystems to climate variability at the University of Arizona is of great interest to me and I believe I can apply my skills to help elucidate the underlying processes.

A key question I have addressed is the effect of disturbance on my field site. I have been based at a flux tower site in southern England with a 15 year record of eddy covariance data. The site's canopy is predominantly oak and the disturbance came when half of the site was thinned under the management regime. I was interested in how we could use the combination of both observations and models in data assimilation to better understand the effect this disturbance had on the forest. A previous statistical analysis suggested that thinning had no effect on the net ecosystem exchange of CO\(_{2}\) at the site. Two competing hypotheses to explain this are that either this could be due to regrowth and increased productivity from reduced competition and higher light availability, or decreased heterotrophic respiration due to reduced priming of soil microbes with starchy carbohydrates. Initial results suggest that the second hypothesis is correct, but I am still in the process of finishing this work. As part of this work I conducted an extensive field campaign to measure the leaf area index of the both the thinned and unthnnied sides of the site. I did this by establishing three transects and using a ceptometer, hemispherical photographs and litter traps to sample at varying spatial resolutions along these transects. I would be very interested in continuing research into the effect of disturbance on ecosystems. At this field site I took observations of a large number of variables relevant to the carbon balance of the forest. This included soil, litter and stem respiration using a infrared gas analyser, leaf are index and diameter at breast height. I also processed the long record of flux and meteorological data at the site. 

I find the research outlined very exciting as I continue to develop my own interests in this field of science. I fit the required profile for the advertised project. From the outlined desired qualifications:
\begin{itemize}
\item On my project I have worked with a very interdisciplinary team. Of my 5 supervisors there are; 3 mathematicians with an interest in data assimilation and numerical weather prediction, an ecologist who is head of a unit within a government run research organisation (Forest Research) and a carbon-cycle scientist. I have also worked closely with another 2 members of staff at Forest Research who are involved in field work and LiDar scanning. Managing the differing input from 7 people has been a skill I have developed. The varied input of the people with whom I work has made my PhD project stronger.
\item I have both strong organisational and communication skills, this can be seen from the attached paper and also giving multiple presentations of both posters and talks at conferences (including an oral presentation at EGU 2016) and within my department. 
\item I have worked extensively with Python for analysis and processing of data from the research site where I am based. I also attended the Flux Course, Colorado in 2014 where I gained an exposure to other data types such as remote sensing data and data from leaf level photosynthesis. Although I do not handle satellite data directly, I am based in Associate Professor Tristan Quaife's group at Reading University who do a lot of work with remote sensing data. 
\end{itemize}

\thispagestyle{empty}
\bibliography{../PhD}{}
\end{document}