\documentclass[11pt]{article}
\usepackage[sort]{natbib}
\usepackage{bm,amsmath,bbm,amsfonts,nicefrac,latexsym,amsmath,amsfonts,amsbsy,amscd,amsxtra,amsgen,amsopn,bbm,amsthm,amssymb,graphicx}
\usepackage{fancyhdr}
\usepackage[margin=0.8in]{geometry}
\bibliographystyle{abbrvnat}

\title{Statement of research interests}
\author{Ewan Pinnington}
\date{}

\newtheorem{theorem}{Theorem}[section]
\newtheorem*{defn}{Definition}


\begin{document}

\maketitle

During my PhD project I have developed interests in modelling and data assimilation for the terrestrial carbon cycle. Data assimilation is a mathematical technique for combining observations with a mathematical model in order to improve our estimate to the state of a system. This technique requires us to quantify the uncertainty in observations and the uncertainty in prior model predictions. Much of my work has been to improve this representation of uncertainty in data assimilation schemes, to both improve the accuracy of posterior model estimates and forecasts of forest carbon balance. The development of data assimilation techniques and comparison of modelled and observed estimates for the terrestrial carbon cycle is something I would like to continue to pursue. For this reason the project is of great interest to me as there is scope for the inter-comparison of both modelled and observed estimates to the sensitivity of North American ecosystems to climate variability.

As part of my project I was based at a forest research station in southern England. Here, I spent a lot of time making observations myself and working with a long record of eddy-covariance data from the site's flux tower. Another question I have tried to address was what was the effect of disturbance on this research site. The site's canopy is predominantly oak and the disturbance came when half of the site was thinned and the other half left unmanaged. I was interested in how we could use the combination of both observations and models in data assimilation to better understand the effect this disturbance had on the forest. It had previously been found that thinning had no effect on the net ecosystem exchange of CO2 of the forest site. We hypothesised that this could be either due to regrowth and increased productivity from reduced competition and higher light availability, or decreased heterotrophic respiration because of the fewer trees priming the soil with sugars. From initial results it appears that our second hypothesis is most correct, but I am still in the process of finishing this work. Continuing research into the effect of disturbance on ecosystems is something I would be very interested in.   

I believe I fit the required profile stated in the advertised project and the research outlined is very exciting to me as I continue to develop my own interests in this field of science. From the outlined desired qualifications:
\begin{itemize}
\item On my project I have worked with a very interdisciplinary team. Of my 5 supervisors there are; 3 mathematicians with an interest in data assimilation and numerical weather prediction, an ecologist who is head of a government run research organisation (Forest Research) and a carbon-cycle scientist. I have also worked closely with another 2 members of staff at Forest Research who are involved in field work and LiDar scanning. Managing the differing input from 7 people has been a skill I have had to develop. I believe the varied input of the people with whom I work has made my PhD project stronger.
\item I believe I can demonstrate both strong organisational and communication skills, this can be seen from the attached paper. I have also given multiple presentations of both posters and talks at conferences and within my department. 
\item I have worked extensively with Python for analysis and processing of data from the research site where I am based. Although I do not have any direct experience of analysis of satellite data I am based in Dr. Tristan Quaife's group at Reading University who do a lot of work with remote-sensing data and so have some familiarity with this. 
\end{itemize}
\bibliography{../PhD}{}
\end{document}