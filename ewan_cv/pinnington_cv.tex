\documentclass[11pt]{article}
\usepackage[sort]{natbib}
\usepackage{bm,amsmath,bbm,amsfonts,nicefrac,latexsym,amsmath,amsfonts,amsbsy,amscd,amsxtra,amsgen,amsopn,bbm,amsthm,amssymb,graphicx}
\usepackage{fancyhdr}
\usepackage[margin=0.8in]{geometry}
\bibliographystyle{abbrvnat}
\usepackage{titlesec}
\usepackage{bibentry}
\bibliographystyle{plainnat}

\titlespacing\section{-5pt}{12pt plus 4pt minus 2pt}{5pt plus 2pt minus 2pt}
\titlespacing\subsection{-5pt}{12pt plus 4pt minus 2pt}{2pt plus 2pt minus 2pt}
\titlespacing\subsubsection{-5pt}{12pt plus 4pt minus 2pt}{0pt plus 2pt minus 2pt}


\title{\vspace{-2cm}Ewan Pinnington}
\author{email: ewan.pinnington@gmail.com}
\date{}

\newtheorem{theorem}{Theorem}[section]
\newtheorem*{defn}{Definition}


\begin{document}

\nobibliography*

\maketitle
 \vspace{-1.cm}
\section*{PhD project}
\subsection*{Reading University (2013 - Present) }
\begin{itemize}
\item \textbf{Title}: Understanding the information content in diverse observations of forest carbon stocks and fluxes for data assimilation and ecological monitoring.
\item \textbf{Description}: Improvement of model predictions of forest carbon balance by the implementation and improvement of data assimilation techniques. Specifically I examined the carbon balance of my research site following disturbance after thinning from management practices. Data assimilation is a mathematical technique for combining observations with a mathematical model to improve model forecasts. I developed techniques to better understand and represent information from both models and observations in data assimilation schemes, for example the errors in eddy-covariance data. As part of the project I was based at a flux tower site in southern England, which has a 15 year record of flux data.
\end{itemize}
\subsection*{Skills developed}
\begin{itemize}
\item \textbf{Programming}: Much of my experience of coding has been with Python but I also have some familiarity with Matlab and Fortran. In the time on my PhD project I have demonstrated on three Python software courses. I am confident both with functional and object oriented programming in Python. I also have experience working with large datasets in Python in many formats including NetCDF. 
\item \textbf{Version control}: During my PhD I have used Git version control software to track and commit all the changes I have made to code I have written and also host my code on GitHub including docummentation. 
\item \textbf{Fieldwork}: I have conducted an extensive fieldwork campaign to measure leaf area index using a ceptometer, hemispherical photographs and regularly sampling litter traps at a forest research station in southern England. During my time based at the research station, I have familiarity in measurements of soil, litter and stem respiration using a LI-COR infrared gas analyser. I also made measurements of diameter at breast height from which I derived biomass via an allometric equation and worked extensively with eddy covariance data from the flux tower at the site. 
\item \textbf{Modelling and data assimilation}: I have coded my own version of the ecosystem carbon balance model DALEC in Python, including tools to find the derivative of the model code via automatic differentiation. I have also implemented my own data assimilation routine with this ecosystem model, using the technique of Four-Dimensional Variational data assimilation (4D-Var). This involved processing a large amount of data including field observations and meteorology to combine with and drive the DALEC ecosystem model.
\item \textbf{Communication}: I have good communication skills having given a number of presentations. I gave an oral presentation at the 2016 EGU general assembly in Vienna for the ``Developments in terrestrial biogeochemical models using model-data integration'' working group. I have also given oral presentations for the Royal Meteorological Society and reached the finals of Reading University's three-minute thesis competition.
\item \textbf{Collaboration}: The research site I am based at is run by the industrial partner on my project, Forest Research. I have been lucky to help contribute to ongoing work at Forest Research. I have produced results using the modelling system I have set up in collaboration with members of staff at Forest Research. This work has been presented as a poster at EGU 2016 and as an oral presentation at the Integrated Carbon Observation System (ICOS) science meeting 2016. 
\end{itemize}

\section*{Qualifications}
\subsection*{Reading University (2009 - 2013)}
\subsubsection*{Mathematics BSc Hons, \(1^{\text{st}}\)}
\begin{itemize}
\item \textbf{Average grades}: Year 1 - 75\%, Year 2 - 80\%, Year 3 - 82\%.
\item \textbf{Awards}: Received the award for excellent achievement in both year 2 and 3 of my degree.
\item \textbf{Modules included}: Analysis, Algebra, Communicating Mathematics, Cryptography, Mathematics for the Digital Economy and Numerical Analysis.
\item \textbf{Skills developed}: Analytic and lateral thinking, teamwork, mentoring, problem solving, programming, presenting, time and personal management.
\end{itemize}
\subsubsection*{Science Foundation Year}
\begin{itemize}
\item (Mathematics, Physics, Chemistry) at A-level standard. \(1^{\text{st}}\) overall with a high \(1^{\text{st}}\) in Mathematics.
\end{itemize}
\subsection*{Sandy Upper School (2002 - 2007)}
\begin{itemize}
\item \textbf{A-levels}: Mathematics (C), Drama (B), Communication studies (B)
\item Awarded the Sherwood Cup for contribution to Drama.
\item 10 GCSE's including Maths (A), English (B) and Double Science (AA) 
\end{itemize}
\subsection*{Driving Licence}
\begin{itemize}
\item Clean driving licence held since December 2006.
\end{itemize}

\section*{Work experience}
\subsubsection*{Personal Tutor (2010 - 2011)}
Gave personal tuition to a student studying A-level Mathematics. This helped me significantly improve my ability to communicate advanced mathematical ideas to someone without previous experience of them. I found this very rewarding, helping the student to pass their exams.    
\subsubsection*{Global Talent Publishing (2007 - 2009)}
As front man for a band, I signed a three year publishing deal with Global Talent (www.thisisglobal.com). The group were given advance payments for song writing and for performances in the UK and USA. I developed confidence and presentational skills performing on stage in front of many different audiences. I gained good life experience travelling America whilst performing there.
\subsubsection*{First Choice Recruitment (2008 - 2010)}
Undertook a number of varied temporary assignments; Warehouse work, Catering and hospitality and Landscape gardening. I gained excellent communication skills whilst working with a large variety of people from different backgrounds, as I undertook different temporary assignments. I became able to adapt quickly to different environments and tasks set.

%\section*{Publications}
%\bibentry{Pinnington2016299}
\section*{Interests and activities}
\begin{itemize}
\item \textbf{Travelling}: I enjoy seeing the world and hope to do so more in the future. I have travelled parts of America and a lot of Europe.
\item \textbf{Music}: I am extremely interested in music having played guitar and been in bands for a large part of my life. My most successful band having been signed as mentioned in the previous section.
\item \textbf{Sport}: I follow football supporting Aston Villa. I am a keen hiker and enjoy heading outdoors for long walks and camping. I also like to keep fit by regularly going to the gym.
\item \textbf{Cooking}: I am an enthusiastic amateur chef and enjoy making a variety of different dishes.
\end{itemize}
%\bibliography{../PhD}{}

\end{document}